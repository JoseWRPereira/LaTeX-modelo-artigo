
\documentclass[
	% -- opções da classe memoir --
	article,			% indica que é um artigo acadêmico
	11pt,				% tamanho da fonte
	oneside,			% para impressão apenas no recto. Oposto a twoside
	a4paper,			% tamanho do papel. 
	% -- opções da classe abntex2 --
	%chapter=TITLE,		% títulos de capítulos convertidos em letras maiúsculas
	%section=TITLE,		% títulos de seções convertidos em letras maiúsculas
	%subsection=TITLE,	% títulos de subseções convertidos em letras maiúsculas
	%subsubsection=TITLE % títulos de subsubseções convertidos em letras maiúsculas
	% -- opções do pacote babel --
	english,			% idioma adicional para hifenização
	brazil,				% o último idioma é o principal do documento
	sumario=tradicional
	]{abntex2}


% ---
% PACOTES
% ---
\usepackage{artigo-modelo}

% --- Informações de dados para CAPA e FOLHA DE ROSTO ---
\titulo{Modelo Canônico de Artigo científico com \abnTeX}
\tituloestrangeiro{Canonical article template in \abnTeX: optional foreign title}

\autor{
Equipe \abnTeX\thanks{``Recomenda-se que os dados de vinculação e
endereço constem em nota, com sistema de chamada próprio, diferente do sistema 
adotado para citações no texto.'' \url{http://www.abntex.net.br/}} 
\\[0.5cm] 
Lauro César Araujo\thanks{``Constar currículo sucinto de cada autor, com 
vinculação corporativa e endereço de contato.''} }

\local{Brasil}
\data{2018, v-1.9.7}
% ---


% ---
% compila o indice
% ---
\makeindex
% ---



% ----
% Início do documento
% ----
\begin{document}

% Seleciona o idioma do documento (conforme pacotes do babel)
%\selectlanguage{english}
\selectlanguage{brazil}

% Retira espaço extra obsoleto entre as frases.
\frenchspacing 

% ----------------------------------------------------------
% ELEMENTOS PRÉ-TEXTUAIS
% ----------------------------------------------------------

%---
%
% Se desejar escrever o artigo em duas colunas, descomente a linha abaixo
% e a linha com o texto ``FIM DE ARTIGO EM DUAS COLUNAS''.
% \twocolumn[    		% INICIO DE ARTIGO EM DUAS COLUNAS
%
%---

% página de titulo principal (obrigatório)
\maketitle


% titulo em outro idioma (opcional)



% resumo em português
\begin{resumoumacoluna}
	\input{resumo}
\end{resumoumacoluna}


% resumo em inglês
\renewcommand{\resumoname}{Abstract}
\begin{resumoumacoluna}
 \begin{otherlanguage*}{english}
 	   According to ABNT NBR 6022:2018, an abstract in foreign language is optional.

\vspace{\onelineskip}

\noindent
\textbf{Keywords}: latex. abntex.

 \end{otherlanguage*}  
\end{resumoumacoluna}

% ]  				% FIM DE ARTIGO EM DUAS COLUNAS
% ---

%\begin{center}\smaller
%\textbf{Data de submissão e aprovação}: elemento obrigatório. Indicar dia, mês e ano
%
%\textbf{Identificação e disponibilidade}: elemento opcional. Pode ser indicado 
%o endereço eletrônico, DOI, suportes e outras informações relativas ao acesso.
%\end{center}

% ----------------------------------------------------------
% ELEMENTOS TEXTUAIS
% ----------------------------------------------------------
\textual

% ----------------------------------------------------------
% Introdução
% ----------------------------------------------------------
\input{secao-introducao}
% ----------------------------------------------------------
% Seção de explicações
% ----------------------------------------------------------
\input{secao-exemplos_de_comandos}
\input{secao-cabecalhos_e_rodapes_customizados}
\input{secao-consulte_o_manual}


% ---
% Finaliza a parte no bookmark do PDF, para que se inicie o bookmark na raiz
% ---
\bookmarksetup{startatroot}% 
% ---

% ---
% Conclusão
% ---
\input{secao-consideracoes_finais}

% ----------------------------------------------------------
% ELEMENTOS PÓS-TEXTUAIS
% ----------------------------------------------------------
\postextual

% ----------------------------------------------------------
% Referências bibliográficas
% ----------------------------------------------------------
%\bibliography{abntex2-modelo-references}
\bibliography{referencias}
% ----------------------------------------------------------
% Glossário
% ----------------------------------------------------------
%
% Há diversas soluções prontas para glossário em LaTeX. 
% Consulte o manual do abnTeX2 para obter sugestões.
%
%\glossary

% ----------------------------------------------------------
% Apêndices
% ----------------------------------------------------------

% ---
% Inicia os apêndices
% ---
\begin{apendicesenv}
\input{apendiceA}
\end{apendicesenv}
% ---

% ----------------------------------------------------------
% Anexos
% ----------------------------------------------------------
\cftinserthook{toc}{AAA}
% ---
% Inicia os anexos
% ---
%\anexos
\begin{anexosenv}

% ---
\input{anexoA}
\end{anexosenv}

% ----------------------------------------------------------
% Agradecimentos
% ----------------------------------------------------------

\input{secao-agradecimentos}

\end{document}
